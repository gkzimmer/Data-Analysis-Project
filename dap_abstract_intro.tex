\section{Abstract}

The process of heavy element production in our universe, known as the neutron capture process, occurs when heavy nuclei are assembled due to the collection of protons and neutrons. The n-capture process can be further split into two extremes- the slow process, which occurs over billions of years, and the rapid process, which occurs in some of the most brief and extreme events in our universe. By analyzing spectral lines from red giant stars in our galactic halo, we can estimate heavy element contributions from the s- and r-process. As these red giant field stars are some of the oldest stars in our galaxy, their abundances of heavy elements can be used to estimate the overall age of our galaxy, giving a better understanding of our evolution as a whole.

\section{Introduction}

The n-capture process occurs only in heavy elements past the iron peak, allowing for the production of rare elements that could not be constructed otherwise. The two divisions of the n-capture process further defines the production of these heavy elements.

The s-process occurs in low to intermediate mass stars that, due to their very nature, have not yet reached the end of their incredibly long lifetimes. As the star undergoes its red giant phase, slowly fizzling out over billions of years, the nuclei have time to slowly capture more and more neutrons. In addition, the long lifetime allows for beta decay to occur, allowing for special isotopes only attainable through the s-process \cite{ncs99}.

In constrast, the r-process occurs due to some of the most violent events in our universe. In this proess, the nuclei are quickly bombarded by neutrons until the nuclei is 'dripping' with them. This process is so brief that there is no time for beta decay to occur, unlike the s-process. The most commonly accepted site for the r-process is in the core collapse of massive stars during the supernova process \cite{ncs02}. Although a majority of the r-process heavy element production has been detected in these supernovae, new detection methods for violent events, such as gravitational waves and gamma-ray bursts, have led to new proposals for r-process sites \cite{mmo17}. One new proposed site for the r-process is in the decompression of the crust of neutron stars, occuring due to violent events such as a neutron star merger or the coalescence of a neutron star and a black hole \cite{rpro07}. These observations have been noted in recent studies, such as in the binary neutron star merger of AT 2017gfo in 2017.

The s- and r-process go hand-in-hand in creating heavy elements, with some elements gaining more from one process over the other, and some elements gaining entirely from only one process. This relationship can be used to estimate overall galactic abundances of certain heavy elements, and, in turn, give a rough estimation of our galaxy's age.

In our research, we have targeted red giant field stars within our own galaxy, utilizing the Milky Way's relationship between star metallicity and age. These red giant field stars reside in our galactic halo and are slowly cooling off over a period of billions of years. Due to their aging process and location on the main sequence track, these stars have puffed out over time, giving themselves a distended atmosphere. These factors, coupled with their isolated location on the fringe of our galaxy, make the red giant field stars perfect candidates for clear spectral data gathering. We have targeted a list of these red giant field stars using pre-existing data from the W.M. Keck Observatory, and then furthermore targeted field lines associated with three heavy elements: Barium, Europium, and Lanthanum. 

All information, code, and data sets are avaliable at the GitHub repository for this project \cite{github}.
